\documentclass{article}

\usepackage[utf8]{inputenc}
\usepackage[T1]{fontenc}
\usepackage[francais]{babel}
\usepackage{fullpage}


\usepackage{amsthm}
\usepackage{amsmath}
\usepackage{amssymb}
\usepackage{mathrsfs}

\title{\textbf{Fouine}}
\date{Mars 2018}
\author{Victor Boone - Gabrielle Pauvert}

\usepackage{graphicx}
\DeclareGraphicsExtensions{.png}
\usepackage{hyperref}

\newcommand\code[1]{{\fontfamily{lmtt}\selectfont #1}}

\usepackage{minted} 
 
\begin{document}

\maketitle
\tableofcontents

\section{Généralités}

	Dans notre implémentation de \emph{Fouine}, il y a $4$ grands types
	
	\vspace{0.5cm}
	
	\begin{tabular}{l l l}
	Composante & Description & Type \\
	\hline
	Expression & Ce qu'on évalue & \code{expr\_f} \\
	Valeurs & Ensemble des valeurs pouvant être renvoyées & \code{val\_f} \\
	Environnement & Stocke des couples (variables, valeurs) & \code{env\_f} \\
	Mémoire & Représentation de la mémoire en Fouine & \code{mem\_f} 
	\end{tabular}
	
	\vspace{0.5cm}
	
	On ne parlera pas de \emph{programme} en Fouine, mais plutôt d'\emph{expression}. Tout est expression qu'on cherche à évaluer. L'évaluation des fonctions est faite par la fonction \code{eval}:
	
	
	\begin{minted}{ocaml}
	
val eval : expr_f -> env_f -> (val_f -> val_f) -> (val_f -> val_f) list -> val_f
	
	\end{minted}
	
	utilisée de la forme \code{eval expr env k kE}, et implémentée par continuations.
	La mémoire \code{mem\_f} est globale.

\subsection{Expressions}

	Les expressions regroupent tout ce qui a été demandé dans le sujet. Leur type est :
	
	\begin{minted}{ocaml}
	
type expr_f =
  | Cst    of int                          (* Feuille : constante *)
  | Bool   of bool                         (* Feuille : booléen  *)
  | Var    of var_f                        (* Feuille : variable *)
  | Bang   of expr_f                       (* Feuille : le déréférençage *)
  | Unit                                   (* Feuille : le type unit *)
  | Pair   of expr_f * expr_f              (* Un couple d'expressions *)
  | Neg    of expr_f                       (* Négation de Booléens *)
  | Bin    of expr_f * operator_f * expr_f (* opérations binaires *)
  | PrInt  of expr_f                       (* built-in prInt *)
  | Let    of pattern_f * expr_f * expr_f  (* let <var_f> = <expr_f> in <exec_f>   *)
  | LetRec of var_f   * expr_f * expr_f    (* let rec *)
  | Match  of expr_f  * pmatch_f           (* match [expr_f] with [pattern_matching] *)
  | IfElse of expr_f * expr_f * expr_f     (* If .. then .. else *)
  | Fun    of pattern_f * expr_f           (* car les fonctions sont un objet fun var -> expr *)
  | App    of expr_f * expr_f              (* Ce sont les applications *)
  | Aff    of expr_f * expr_f              (* Affectation i.e le `:=`*)
  | Alloc  of expr_f                       (* Allocation mémoire *)
  | Try    of expr_f * var_f * expr_f      (* Le 'try ... with E ... -> g...' *)
  | Raise  of expr_f                       (* raise E ... : qui sera un int en pratique *)


	\end{minted}
	
	\vspace{0.5cm}
	Voici un tableau de correspondance entre les constructeurs et les notions OCamL.
	
	\begin{center}
	\begin{tabular}{l | l}
		Constructeur & Équivalent OCamL \\
		\hline
		\code{Var of var\_f} & \code{x, y, c0, variable\_1} : nom de variable \\
  		\code{Bang of expr\_f} & \code{!} : déréférençage\\
  		\code{Bool of bool} & \code{true}, \code{false} : les booléens\\
		\code{Cst of int} & \code{0}, \code{1} : les entiers\\
		\code{Neg of expr\_f} & \code{not ...} : négation booléenne \\
		\code{Bin of expr\_f * operator\_f * expr\_f} & \code{x+5} : opérations binaires (\code{+, -, *, /, mod, ||, <, =, ...}) \\
  		\code{PrInt of expr\_f} & \code{let prInt x = print\_int x; print\_newline ()} \\
		\code{Let of pattern\_f * expr\_f * expr\_f} & \code{let ... = ... in ... } \\
		\code{LetRec of var\_f * expr\_f * expr\_f} & \code{let rec ... = ... in ... }\\
		\code{IfElse of expr\_f * expr\_f * expr\_f} & \code{if ... then ... else ...} \\
		\code{Fun of pattern\_f * expr\_f} & \code{fun x -> ... } \\
		\code{App of expr\_f * expr\_f} & \code{ a b } : application \\
		\code{Aff of expr\_f * expr\_f} & \code{... := ...} \\
  		\code{Alloc of expr\_f} & \code{ref ...} : allocation mémoire \\
  		\code{Pair of expr\_f * expr\_f} & \code{... , ...} : couples \\
  		\code{Raise of expr\_f } & \code{raise (E ...) } : levée d'exceptions \code{E}\\
  		\code{Try of expr\_f * var\_f * expr\_f} & \code{try ... with E x -> ...} 
	\end{tabular}
	\end{center}
	
	\vspace{0.5cm}
	
	Certains constructeurs intermédiaires sont détaillés dans \code{type.ml}.
	Les booléens font leur apparition dans les expressions de ce \emph{Fouine}, principalement pour simplifier les transformations de programmes. \code{Bin} a été étendu en conséquence.

\subsection{Valeurs}


	Les programmes fouine sont des expressions \code{expr\_f}, et sont évaluées par la fonction \code{eval}. Une évaluation renvoie une valeur du type donné ci-dessous :
\begin{minted}{ocaml}
type val_f = Unit
          | Bool        of bool
          | Int         of int
          | Ref         of int
          | Cons        of string * val_f
          | Pair_val    of val_f * val_f
          | Fun_val     of pattern_f * expr_f * env_f
\end{minted}

	Concernant les fonctions, celles-ci sont de la forme \code{fun x -> expr}. On sauvegarde de plus l'environnement dans lequel elles ont été définies (notion de clotûre). Nous y reviendrons. \code{Cons} n'est pas utilisé.


\subsection{Environnement}

	L'environnement est une liste d'association (variable, valeur), qui agit comme une \textbf{pile} (on empile les associations les unes après les autres). Une variable est simplement une chaîne de caractères. Ainsi :
	
	\begin{minted}{ocaml}
		type env_f = (var_f * val_f) list
	\end{minted}
	
	Si \code{x} est une variable, sa valeur associée dans un environnement \code{env} est la première occurrence \code{("x", ...)} dans l'environnement. Donc, si on considère la liste \code{l = [("x", Int(0)); ("x", Int(5))]}, la valeur de \code{x} courante est \code{Int(0)}.
	
	Si on essaie de lire la valeur d'une variable non-existante, l'interpréteur lève une exception \code{Failure}.

\subsection{Mémoire}

	
	La mémoire est un tableau de valeurs de taille fixe \code{1000000}. Ainsi, on impose à \emph{Fouine} une quantité bornée de mémoire. Si celle-ci est dépassée, l'interpréteur lève \code{Failure "Out of Memory"}. Les cases mémoires étant typées \code{val\_f}, elles peuvent occuper une taille non bornée de mémoire.
	
	\begin{minted}{ocaml}
let mem = Array.make 1000000 Unit
let available = ref 0;;
	\end{minted}
	
	\vspace{0.5cm}

	C'est la mémoire qui gère les références. Pour les gérer, on descend assez bas-niveau dans la philosophie. Les références seront vues comme une adresse mémoire. Ainsi, \code{let a = ref 0} est interprété comme

	\vspace{0.5cm}

\begin{itemize}
	\item Allouer une nouvelle case mémoire
	\item Associer \code{a} à cette nouvelle case \emph{(notion d'adresse mémoire)}
	\item Mettre le contenu de cette case à $0$
\end{itemize}

	\vspace{0.5cm}
	
	Pour savoir qu'elle est la prochaine case mémoire allouable, on utilise une variable globale \code{available}.

\section{Parsing}


\subsection{\code{let ... in}}

Ce qui pose problème avec les \code{let ... in ...} c'est qu'il y en a beaucoup de variantes! On distingue les \code{let} des expressions des \code{let} extérieurs (du toplevel) qui permettent une syntaxe un peu particulière pour enchaîner les \code{let} sans \code{in} ni \code{;;}. Il y a aussi l'ajout (parfois facultatif) du mot clé \code{rec} qui dédouble tous les cas. Enfin les \code{let} peuvent servir à définir des fonctions ou des couples (triplets, etc.). Les let ont été un peu regroupés depuis le rendu 3. Une variable simple est vue comme un cas particulier de uplet à un élément. On définit une règle \code{var\_pattern} qui admet tous ces cas. Les \code{let rec} ne sont pas concernés: on ne peut définir qu'au plus une variable avec un \code{let rec}.

\subsection{L'application de fonction}

Typiquement, une application de fonctions est du type \code{expression expression}, mais écrire cela en tant que règle aurait abouti à du "rule never reduce" car on peut toujours lire une expression derrière, et choisir de \code{shift}. Le plus simple était d'expliciter tous les cas dans une règle \code{applicator applicated} où \code{applicator} sert à enchaîner les applications (curryfication) et \code{applicated} détaille les possibilités de fonctions et d'arguments.

Pour la définition de fonction, il a fallu faire attention aux différentes façons de définir une fonction (avec des \code{fun ->} ou directement, ce qui complique encore un peu les définitions de \code{let... in...}).


\section{Evaluation}

\subsection{Style d'évaluation}

	L'évaluation se fait par continuation : \code{eval expr env k kE}. La fonction d'évaluation reçoit quatre arguments :

	\begin{itemize}
		\item \code{expr} l'expression à évaluer
		\item \code{env}  l'environnement courant (définition des variables)
		\item \code{k}	  la continuation courante
		\item \code{kE}	  la pile de continuations d'exceptions	
	\end{itemize}
	
	Nous renvoyons au code pour l'implémentation de \code{eval}. Certains passages du code sont plus lourd car \code{eval} est capable d'afficher l'expression courante en cas d'erreur - l'implémentation de l'évaluation fainéante a aussi été gourmande en lignes de code.
	
\subsection{Fonctions}

	\subsubsection{Fonctions classiques}

	Les fonctions sont définies avec une clôture pour la raison suivante 
	
	\begin{minted}{ocaml}
let a = 5 in
let f = fun x -> a in
let a = 10 in
f 10;;
	\end{minted}
	
	Ici, \code{f} doit se souvenir de la valeur de \code{a}. Sauvegarder une clôture est en ${\cal{O}}(1)$ : il s'agit juste de sauvegarder un pointeur vers une tête de pile i.e l'environnement courant. La valeur d'une fonction est 
	
	\begin{minted}{ocaml}
	
	| Fun_var of var_f * expr_f * env_f 
	
	\end{minted}
	
	donc les fonctions ne sont pas typées en fouine.
	
	\subsubsection{Fonctions Récursives}	

	Quand on définit une fonction \code{f} de la manière ci-dessus, il faut remarquer qu'au moment de sa définition, \code{f} n'est pas définie dans l'environnement, donc une fonction standard n'est pas définie dans sa propre clôture. On ne peut donc pas définir des fonctions récursives. On introduit le \code{let rec}.
	
	\vspace{0.5cm}
	
	Le \code{let rec} est un peu particulier. Il regarde si on est en train de définir une fonction ou non. Si oui, il construit la clôture de \code{f} en y rajoutant l'association \code{(f, clôture où f est défini)}. Il y a donc une définition cyclique ici, qui fera apparaître \code{<cycle>}. Un détail important ; lors d'un
	
	\begin{minted}{ocaml}
let rec f = e1 in e2
  \end{minted}
  
  on commence par évaluer \code{e1}. Il y a alors deux possibilitées. Soit on reçoit une valeur fonctionnelle \code{Fun\_val("f", e0, env0)}, avec \code{env0} une clôture dans laquelle \code{f} n'est pas défini. C'est là qu'on utilise le \code{let rec} d'OCamL pour redéfinir la clôture de la fonction avant de rajouter \code{f} dans l'environnement. On évalue ensuite \code{e2}.
	
	\vspace{0.5cm}
	
	Si c'est une variable quelconque, qui n'est pas une fonction, on se comporte comme un \code{let} classique. Ceci ne pose pas trop de problèmes si on ne joue qu'avec des \code{int} ou des couples, mais fait tout de même apparaître quelques petits cas pathologiques (cf dernière partie).



\section{Exceptions}

  \code{eval} est implémenté par continuations, et dispose d'une pile de continuations d'exceptions. En cas de \code{try e1 with e2} l'interpréteur ajoute une nouvelle continuation d'exception sur la pile pour l'évaluation de \code{e1}. Quant à \code{raise e1}, il évalue \code{e1}, puis dépile la pile de continuations d'exceptions pour traiter le résultat.
  
  \vspace{0.5cm}
  
  \code{Remarque :} On pourrait se débarrasser de la pile, pour n'avoir qu'une continuation d'exception à chaque fois en changeant 
  
  \begin{minted}{ocaml}
| Raise expr ->
		begin
		  match k' with
		  | [] -> failwith "Raise : Nothing to catch exception"
		  | k_exn :: k' -> eval expr env k_exn k'
    end
| Try (expr1, var_except, expr2) ->
   eval expr1 env k ((fun exn -> eval expr2 (env_aff var_except exn env) k k') :: k')
  \end{minted}
  
  en
  
  \begin{minted}{ocaml}
| Raise expr ->
	 eval expr1 env k' k'
| Try (expr1, var_except, expr2) ->
   eval expr1 env k (fun exn -> eval expr2 (env_aff var_except exn env) k k')
   \end{minted}
  
  Ceci nous a surpris à première vue, mais c'est ce qu'on a fait pour les transformations de programme et cela semble très bien marcher.

\section{Transformations de Programmes}

	On met ici les formules utilisées pour les transformations de programmes. Ces dernières ont été converties en expressions d'arbres de programme en utilisant directement le \textit{parser}.
	
\subsection{Impératives}

	On rajoute en pratique un \code{\_\_} aux variables utilisées pour différencier de celles du programme initial.
	
	\begin{minted}{ocaml}

[| 42 |] (* devient *) fun s -> (42,s)
[| true |] (* devient *) fun s -> (true,s)
[| x |]  (* devient *) fun s -> (x,s)
[| !e |] (* devient *) fun s -> let (l,s1) = [| e |] s in
                       let v = read s1 l in (v,s1)
[| () |] (* devient *) fun s -> ((),s)

[| (e1, e2) |] (* devient *) 
  fun s -> let (v2,s2) = [| e2 |] s in
           let (v1,s1) = [| e1 |] s2 in
           ((v1,v2),s1)

[| not e0 |] (* devient *) 
  fun s -> let (b,s0) = [| e0 |] in
           (not b, s0)
[| e1 + e2 |] (* devient *) 
  fun s -> let (v2,s2 ) = [[e2]] s in
           let (v1,s1) = [[e1]] s2 in
           (v1 op v2,s1)

[| prInt e0 |] (* devient *) 
  fun s -> let (v0,s0) = [| e0 |] s in
           (prInt v0, s0)
[| let pat = e1 in e2 |]   (* devient *) 
  fun s -> let (pat,s1) = [| e1 |] s in
           [| e2 |] s1
[| let rec v = e1 in e2 |] (* devient *) 
  fun s -> let rec v = (let (f,s0) = [| e1 |] s in f)
           in [| e2 |] s

[| match |] (* non supporté *)
[| if b then e1 else e2 |] (* devient *) 
  fun s -> let (b0, s0) = [| b |] s in
           if b0 then [| e1 |] s0
           else [| e2 |] s0
           
[| fun pat -> e |] (* devient *)  fun s -> ((fun pat -> [| e |]), s)
[| e1 e2 |] (* devient *) 
  fun s -> let (v, s2) = [| e2 |] s  in
           let (f, s1) = [| e1 |] s2 in
           f v s1

[| e1 := e2 |] (* devient *) 
  fun s -> let (l1, s1) = [| e1 |] s in
           let (v2, s2) = [| e2 |] s2 in
           let s3 = write s2 l1 v2 in
           ((), s3)
[| ref e0 |] (* devient *) 
  fun s -> let (v,s1) = [| e0 |] s in
           let (s2,l) = alloc s1 in
           let s3 = write s2 l v in
           (l,s3)
           
[| try e1 with E x -> e2 |] (* devient *) 
  fun s -> try let (v1, s1) = [| e1 |] s in (v1, s1)
           with E x -> [| e2 |] s
           
[| raise e0 |] (* devient *) 
  fun s -> try let (v,s0) = [| e0 |] s in
	       (raise (E v), s0)
	
	\end{minted}
	

\subsection{Continuations}

	On rajoute en pratique un \code{\_} aux variables utilisées pour les différencier de celles du programme initial.

	\begin{minted}{ocaml}

[| () |] (* devient *) fun k kE -> k ()
[| true |] (* devient *) fun k kE -> k true
[| 42 |] (* devient *) fun k kE -> k 42
[| x |]  (* devient *) fun k kE -> k x

[| !e |] (* devient *) 
	fun k kE -> [| e |] (fun addr -> k (!addr)) kE

[| not e |]   (* devient *) 
	fun k kE -> [| e |] (fun b -> k (not b)) kE
[| e1 + e2 |] (* devient *) 
	fun k kE -> [| e2 |] (fun v2 -> [| e1 |] (fun v1 -> k (v1 + v2)) kE) kE
[| prInt e |] (* devient *) 
	fun k kE -> [| e |] (fun v -> k (prInt v)) kE

[| if be then e1 else e2 |] (* devient *) 
	fun k kE -> [| be |] (fun b -> (if b then [| e1 |] else [| e2 |]) k kE) kE
	
	\end{minted}
	
	C'est là qu'il est agréable de supporter les booléens, car \code([| be |]) renvoie une valeur booléenne. Il est possible d'esquiver ce problème autrement, mais de manière moins fluide.
	
	\begin{minted}{ocaml}
	
[| let x = e1 in e2 |] (* devient *) 
	fun k kE -> [| e1 |] (fun v -> let x = v in [| e2 |] k kE) kE 
[| let rec f = e1 in e2 |] (* devient *) 
	fun k kE -> [| e1 |] (fun v -> let rec f = v in [| e2 |] k kE) kE

	\end{minted}
	
	Ici, on utilise le fait que \code{[| e1 |]} renvoie une valeur fonctionnelle, mais non récursive. L'utilisation d'un \code{let rec} juste derrière permet de redéfinir cette valeur fonctionnelle comme étant récursive. Ceci fonctionne car \emph{Fouine} n'évalue pas les fonctions, mais les renvoie directement avec leur clôture.
	
	\begin{minted}{ocaml}
	
[| match |] (* non supporté *)
[| fun x -> e |] (* devient *) 
	fun k kE -> k (fun x -> [| e |]) 
[| e1 e2 |] (* devient *) 
	fun k kE -> [| e2 |] (fun v -> [| e1 |] (fun f -> f v k kE) kE) kE

[| e1 := e2 |] (* devient *) 
	fun k kE -> [| e2 |] (fun v -> [| e1 |] (fun addr -> k (addr := v)) kE) kE
[| ref e |]    (* devient *) 
	fun k kE -> [| e |] (fun v -> k (ref v)) kE

[| (e1, e2) |] (* devient *) 
	fun k kE -> [| e2 |] (fun v2 -> [| e1 |] (fun v1 -> k (v1, v2)) kE) kE	

[| try e1 with E x -> e2 |] (* devient *) 
	fun k kE -> [| e1 |] k (fun x -> [| e2 |] k kE)
[| raise e |]               (* devient *) 
	fun k kE -> [| e |] (fun v -> kE v) kE
	
	
	\end{minted}


\section{La machine SECD}

\subsection{La transformation}

Le code de la machine SECD est représenté par un tableau d'instructions, et ressemble beaucoup à de l'assembleur. Les indices du tableau sont utilisés pour effectuer des \code{JUMP} à des endroits précis du code (utile par exemple pour les conditions et les exceptions). On effectue la transformation en deux étapes: une première passe sert à déterminer le nombre d'instructions que l'on va avoir dans le tableau, ce qui nous permet de créer un \code{array} de la bonne taille, que l'on remplit avec les instructions pendant une deuxième passe.

Pour le cas du let: on prend en argument un uplet (\code{var\_pattern}). A l'aide d'une fonction récursive auxiliaire \code{traite\_pattern}, on ajoute autant de \code{DESTRUCT} puis de \code{LET} qu'il y a de variables dans cet uplet. On garde en mémoire le nombre de \code{LET} mis, puis on ajoute à la fin ce nombre de \code{ENDLET} dans le tableau d'instructions.

Concernant les exceptions, on utilise trois instructions \code{SETJMP}, \code{LONGJMP} et \code{UNSETJMP} qui utilisent une deuxième pile, appelée \emph{pile d'exception}. Cette pile est utile pour sauter dans le code lors de l'instruction \code{LONGJMP} à l'adresse sauvegardée par le dernier \code{SETJMP} (un \emph{savepoint}), et restaurer l'environnement. L'instruction \code{UNSETJMP} permet d'enlever le dernier \emph{savepoint} dans le cas où on a exécuté le corps \code{e1} d'un \code{try e1 with ... -> e2} sans lever d'exception.

\begin{minted}{ocaml}

  Cst a (* devient *) CONST a
  Bool b (* devient *) BOOL b
  Var x (* devient *) ACCESS x
  !e (* devient *) [e]; READ
  () (* devient *) UNIT
  (e1,e2) (* devient *) [e2]; [e1]; PAIR
  not(e) (* devient *) [e]; NOT
  e1 + e2 (* devient *) [e1]; [e2]; ADD
(* de même pour les autres opérations arithmétiques et booléennes *)
(* cas particulier: or et and sont transformés pour l'évaluation fainéante: 
a and b devient if a then b else false
a or b devient if a then true else b *)
  prInt e (* devient *) [e]; PRINT
  let pattern = e1 in e2 (* devient *) [e1]; LET (pattern); [e2]; ENDLET (pattern)
  let rec f = e1 in e2 (* devient *) [e1]; REC f; [e2]; ENDLET
  if b then e1 else e2 (* devient *) [b]; JUMPIF (addr1); [e2]; JUMP (addr2); [e1]
(* où addr1 pointe sur le début de [e1] et addr2 pointe sur la première instruction suivant [e1] *)
  fun pattern -> e (* devient *) CLOSURE (addr); LET (pattern); [e]; ENDLET (pattern); RETURN
(* où addr pointe vers la première instruction suivant RETURN *)
  e1 e2 (* devient *) [e2]; [e1]; APPLY
  e1 := e2 (* devient *) [e2]; [e1]; WRITE
  ref e (* devient *) [e]; ALLOC
  try e with E x -> eX (* devient *) SETJMP (addr1); [e]; UNSETJMP; JUMP (addr2); LET x; [eX]; ENDLET
(* où addr1 pointe vers LET x et addr2 pointe vers la première instruction après le ENDLET *)
  raise e (* devient *) [e]; LONGJMP

\end{minted}

\subsection{L'exécution}

	\paragraph{Types} Les types utilisés par la machines ne peuvent pas être exactement identiques à la fonction d'évaluation, car on a besoin de pouvoir empiler des environnements, des clôtures dont la syntaxe est modifiée, etc. Ceci est permis car on n'a pas à mélanger la machine avec l'interpréteur. Les types utilisés sont définis comme suit :
	
	\begin{minted}{ocaml}
	type mval_f =
  	   INT of int
 	 | BOOL of bool
 	 | UNIT
 	 | ENV  of menv_f
 	 | CLO  of int * menv_f
 	 | ADDR of int
  	 | PTR  of int
 	 | PAIR of mval_f * mval_f
	and  menv_f = (var_f * mval_f) list

	type stack_f = mval_f list
	type xstack_f = int * menv_f * stack_f list
	;;
	\end{minted}
	
	où \code{xstack\_f} correspond à la pile d'exceptions.
	
	
	\paragraph{} On définit ensuite les instructions :
	
	\vspace{0.5cm}

\begin{footnotesize}
\begin{tabular}{ c | c | c  | c || c | c | c }

Mot clé & Env avant & Stack avant & PC avant & Env après & Stack après & PC après \\
\hline
CONST c & e & s & pc & e & c.s & pc+1 \\

BOOL b & e & s & pc & e & b.s & pc+1 \\

ACCESS x & e & s & pc & e & x.s & pc+1 \\

ADD/etc. & e & v1.v2.s & pc & e & v1+v2.s & pc+1\\

NOT & e & b.s & pc & e & not(b).s & pc+1\\

EQ/etc. & e & v1.v2.s & pc & e & (v1 == v2).s & pc+1\\

PRINT & e & v.s & pc & e & v.s & pc+1 \\

LET x & e & v.s & pc & (x,v).e & s & pc+1\\

REC x & e & v.s & pc & (x,v).e & s & pc+1\\

ENDLET & (x,v).e & s & pc & e & s & pc+1\\

JUMP addr & e & s & pc & e & s & addr\\

JUMPIF addr & e & b.s & pc & e & s & addr si b==True et pc+1 sinon\\

CLOSURE addr & e & s & pc & e & clo(pc+1, e).s & addr\\

APPLY & e & f.x.s & pc & e & (f x).s & pc+1\\

RETURN & e & addr.s & pc & e & s & addr\\

READ & e & ptr.s & pc & e & read(ptr).s & pc+1\\

WRITE & e & ptr.v.s & pc & e & ().s & pc+1\\

ALLOC & e & v.s & pc & e & ptr.s & pc+1\\

SETJMP addr & e & s & pc & e & s & pc+1\\

UNSETJMP & e & s & pc & e & s & pc+1\\

LONGJMP & e & s & pc & e & xn.s & ?\\

PAIR & e & l.r.s & pc & e & (l, r).s & pc+1\\

DESTRUCT & e & (l, r).s & pc & e & l.r.s & pc+1\\

\end{tabular}
\end{footnotesize}

	
	\paragraph{} Les instructions \code{SETJMP} \code{LONGJMP} et \code{UNSETJMP} ont un effet sur la pile d'exception, détaillé dans la sous-section précédente. 
	
	
	\paragraph{} Les instructions \code{WRITE}, \code{READ} et \code{ALLOC} ont un effet sur la mémoire de la machine. La mémoire est globale, et de taille bornée par \code{1000000}.

	\begin{itemize}
		\item \code{ALLOC} alloue une nouvelle cas qu'il initialise à la valeur qu'il a dépilé, et empile l'adresse de cette case.
		\item \code{WRITE} écrit dans la case \code{ptr} la valeur \code{v} qu'il a dépilé, puis empile \code{unit}.
		\item \code{READ} empile la valeur lue dans la mémoire à l'adresse dépilée.
	\end{itemize}



\section{Inférence de type et type-checking}

	Les codes relatifs à cette partie sont \code{typeChecker.ml}, \code{checkerTypes.ml}, \code{typePrinter.ml}.
	
	Pour typer : utiliser l'option \code{-type} qui renvoie le type d'une expression. Dans l'exécution sans option, on fait un vérifie les types avant de commencer. On peut mettre l'option \code{-a} pour enlever cette fonctionnalité.
	
	\paragraph{Introduction} L'inférence de type a été l'un des passages les moins triviaux. L'inférence de type est un problème difficile qui est \code{DEXPTIME}-complet d'après Google. Pour le typage, nous nous sommes basé sur le système d'OCamL, qui prend en compte le \emph{polymorphisme}. Le problème de l'inférence de type revient à un problème d'unification. À une expression fouine, on va associer un type que l'on va chercher à unifier selon certaines contraintes. Nous n'avons pas cherché à réinventer la passoire à spaghetti ; l'algorithme au cœur de \emph{type-checker} est l'algorithme de Hindley-Milner.
	
	\paragraph{} Son implémentation n'est pas très performante, car on utilise des substitutions directement. Autrement dit, on explicite une substitution sous la forme d'une liste d'association pour résoudre le problème d'unification, qu'on applique au type "esquissé" de l'expression fouine pour obtenir un type acceptable.
	
	\subsection{Hindley - Milner}

	\paragraph{}Pour les types, on considère les entiers machine \code{int}, les booléens \code{bool}, le type \code{unit}, le type \code{... ref}, ainsi que \code{*} et \code{->}. On prend en compte le polymorphisme (\emph{polymorphism} : \code{'a}) et le polymorphisme faible (\emph{weak polyporphism} : \code{'\_a}) représentés respectivement par \code{Poly\_t} et \code{Weak\_t}.
	
	\begin{minted}{ocaml}
	
	type type_f = 
 	 | Unit_t
 	 | Int_t 
 	 | Bool_t 
 	 | Poly_t  of int
 	 | Weak_t  of int
 	 | Ref_t   of type_f
 	 | Pair_t  of type_f * type_f
 	 | Fun_t   of type_f * type_f
	;;
	\end{minted}
	
	\paragraph{} Lorsque l'on cherche le type d'une expression, on va générer un ensemble de types inconnus en chemin (notamment lors d'un \code{fun x -> ...}, on ne connait pas à l'avance le type de \code{x}). L'objectif de l'algorithme est de typer ces inconnues. Il procède en deux étapes : \textbf{(1)} Générer l'ensemble des contraintes d'unification \textbf{(2)} Unifier l'ensemble des contraintes pour en déduire une \emph{substitution} qui remplacera les inconnues par leur type. 
	
	\paragraph{} Fondamentalement, les inconnues sont les variables. Hindley-Milner fonctionne dans le cas où il n'y a pas de doublon dans les noms de ces dernières. Plutôt que de les renommer pour avoir des noms injectifs dans le code, on simule un environnement comme dans la fonction d'évaluation. On a une liste d'assocation \code{var\_f * int}, où l'entier est un identifiant initialisé lors d'un \code{let}. C'est un indice dans un tableau qui retient les types de tous les inconnus. \textbf{Les inconnues sont toujours représentées en pratique par un type \code{'\_a} pour l'unification}. Ce type est deviné (voire \emph{généralisé}, voir partie suivante) pour être remplacé dans l'expression finale.
	
	\paragraph{Génération des contraintes} Je note ici \code{t(e)} le type de l'expression \code{e}. On procède par induction sur la structure de l'expression fouine fournie. Je laisse le \code{let rec} volontairement de côté pour l'instant.
	\begin{itemize}
	\item \textbf{\code{let x = e1 in e2}} : \code{x} est une nouvelle variable, qu'on rajoute dans l'environnement de type (on crée un nouveau type polymorphique faible pour celle-ci), et on a \code{t(x) = t(e1)} ainsi que \code{t(let x = e1 in e2) = t(e2)}.
	\item \textbf{\code{x}} : on renvoie \code{t(x)}
	\item \textbf{\code{12}} : on renvoie \code{int}
	\item \textbf{\code{e1 + e2}} : alors \code{+} est de type \code{s -> t -> r} et on impose \code{t(e1) = s}, \code{t(e2) = t}, \code{t(e1 + e2) = r}.
	\item \textbf{\code{fun x -> e}} : \code{x} est une nouvelle variable, qu'on associe à un nouveau type \code{'\_a} pour le typage de \code{e}, puis on impose \code{t(fun x -> e) = t(x) -> t(e)}.
	\item \textbf{\code{e1 e2}} : on crée un nouveau type inconnu \code{'\_a} puis on impose \code{t(e1 e2) = '\_a} et \code{t(e1) = t(e2)->'\_a}.
	\item \textbf{\code{if b then e1 else e2}} : on impose simplement \code{t(b) = bool}, ainsi que \code{t(e1) = t(e2)} et \code{t(if b then e1 else e2) = t(e1)}
	\item \textbf{\code{(e1, e2)}} : on renvoie \code{t((e1, e2)) = (t(e1), t(e2))}
	\item \code{!}, \code{ref}, \code{raise} peuvent être vu comme des fonctions, de même que tous les opérateurs unaires et binaires. Quant à \code{try ... with}, son cas est analogue au \code{if then else}.
	\end{itemize}
	
	On peut alors générer un ensemble de contraintes que je note $C$.
	
	\paragraph{Unification} On part de la substitution triviale, représentée par \code{[]} dans le code, puis on considère les éléments de $C$. Notons $s$ la substitution courante. On prend le premier élément de $C$.
	
	\begin{itemize}
	\item \code{t = t} : Alors $s$ est inchangé et on continue.
	\item \code{'\_a = '\_b} : Alors on rajoute à la substitution \code{'\_a <- '\_b}.
	\item \code{'\_a = s -> t} : Si \code{'\_a} n'apparaît ni dans \code{s} ni dans \code{t}, on rajoute à la substitution \code{'\_a <- s -> t}. On procède de même pour \code{ref} et \code{,}. Sinon, on explose en disant qu'on a un cycle.
	\item \code{s -> t = u -> v} : On rajoute dans $C$ les contraintes \code{s = u} et \code{t = v}. De même pour \code{ref} et \code{,}.
	\item Si on essaie d'unifier de symboles non-compatibles, par exemple \code{->} et \code{,}, on explose en disant que les types de correspondent pas. De même si on essaie d'unifier deux types terminaux (\code{int, bool, unit}) différents.
	\end{itemize}
	
	
	\paragraph{Hindley-Milner} Si on sépare clairement la collecte des contraintes de l'unification en deux étapes, on obtient l'algorithme de Hindley-Milner. C'est ce qu'on avait implémenté dans un premier temps. Cet algorithme est très puissant, et exhibe toujours un type le plus général possible ; il permet de typer \code{fun x -> x} en \code{'\_a -> '\_a}, ce qui bien (en supposant que \code{'\_a} est remplacé par du \code{'a}). On peut alors typer des choses "compliquées" telles que :
	
	\begin{minted}{ocaml}
		fun x y z -> x z (y z)
	\end{minted}
	
	en : \code{('\_c -> ('\_e -> '\_f)) -> (('\_c -> '\_e) -> ('\_c -> '\_f))}. Donc on est content. 
	
	
	\paragraph{Limites} Cependant, cet algorithme ne gère pas le "vrai" polymorphisme. En effet, rappelons que le type \code{'\_a} est le type \emph{"je ne sais pas"} et non pas le type \emph{"n'importe quoi"} comme \code{'a}. C'est la différence entre le polymorphisme faible et le polymorphisme : \code{'\_a} prend la "valeur" du premier type qu'il rencontre - ce définitivement, mais \code{'a} est universel. Par exemple, on ne peut pas typer :
	
	\begin{minted}{ocaml}
		let f = fun x -> x in (f 5, f true)
	\end{minted}
	
	Car lors de l'unification, l'algorithme voit \code{f true}, donc pense que \code{f} est de type \code{bool -> 'a}. Puis il voit \code{f 5}, donc on essaie d'unifier \code{int -> 'a} avec \code{bool -> 'a}, donc d'unifier \code{int} avec \code{bool}. Et là, l'algorithme explose en disant que \code{bool} et \code{int} ne sont pas unifiable. Alors qu'on a envie de dire qu'il devrait !
	
	\subsection{Généralisation du \code{let} - code final}
	
	\paragraph{} Une solution à ça s'appelle (en anglais) le \emph{let polymorphism}. L'idée est la suivante, quand on fait \code{let f = fun x -> x}, on définit une fonction \code{'\_a -> '\_a}, donc, comme cette fonction n'impose par de restriction sur \code{'\_a}, celui-ci est en fait un \code{'a}. Dans la littérature, on voit écrit \code{$\forall$a, a -> a}. De notre côté, on utilise le type \code{Poly\_t}.
	
	\paragraph{Polymorphisme du \code{let}} À la fin du typage d'un \code{let}, on va \emph{généraliser} le type obtenu à un type polymorphe, si l'expression de \code{e1} dans \code{let x = e1 in e2} le permet. Par exemple, on généralise si \code{e1} est une fonction, une constante, ou un nom, mais pas si c'est une application, ce qui inclus \code{!}, \code{ref}, et les opérations. On dispose en pratique d'une fonction \code{generalize} qui généralise le type des variables après un \code{let}, et d'une fonction \code{is\_generalizable} qui indique si on a le droit de généraliser une expression.
	
	\paragraph{} Pour pouvoir faire ceci sans problème, il faut qu'on puisse décider du type de \code{x} avant de le généraliser. On ne peut donc plus faire un bel algorithme en deux phases distinctes, en tout cas pas aussi facilement. On a donc choisi de forcer une unification lors d'un \code{let} (ce qui permet de se débarasser de toutes les contraintes courantes), pour pouvoir ensuite \emph{généraliser} le type de \code{x} qui est alors connu.
	
	\paragraph{} Ensuite, lorsqu'on rencontre une variable dans le code, on ne renvoie pas son type, mais on l'\emph{instancie} (i.e on ne dit pas \code{t(x)} mais plutôt \code{t'(x)}). C'est-à-dire qu'on crée un nouveau type isomorphe à \code{t(x)}, qui permettra lors de l'utilisation de \code{x} de ne pas mettre des restrictions sur son type - ce qui peut poser problème lors de l'unification comme on vient de le voir. Par exemple, dans le cas de :
	
	\begin{minted}{ocaml}
		let f = fun x -> x in (f 5, f true)
	\end{minted}
	
	\begin{itemize}
		\item On type d'abord \code{f} en \code{'\_a -> '\_a}
		\item Ce type est généralisé par le \code{let} en \code{'a -> 'a}
		\item On rencontre \code{f true}. En typant \code{f}, on remarque que ce dernier possède le type polymorphe \code{'a}. On crée donc un nouveau type \code{'\_c} pour le représenter i.e \code{t'(f) = '\_c -> '\_c}. On unifie alors \code{'\_c -> '\_c} et un \code{bool -> '\_b}. On en déduit : \code{bool = '\_c}, \code{'\_b = '\_c = bool}. Donc \code{f true} est de type \code{bool} et il n'y a pas de problème de typage.
		\item Puis, pareillement : on rencontre \code{f 5}. En typant \code{f}, on remarque que ce dernier possède le type polymorphe \code{'a}. On crée donc un nouveau type \code{'\_e} pour le représenter i.e \code{t'(f) = '\_e -> '\_e}. On unifie alors \code{'\_e -> '\_e} et un \code{int -> '\_d}. On en déduit : \code{int = '\_e}, \code{'\_d = '\_e = int}. Donc \code{f 5} est de type \code{int} et il n'y a pas de problème de typage.
		\item On conclut en renvoyant le type \code{(int * bool)}
	\end{itemize}
	
	
	\subsection{Typage de fouine}
	
	\paragraph{Récursivité} Pour les fonctions récursives : \code{let rec f = e1 in e2}, leur type est sensé être défini dans \code{e1}. Dans les faits, on commence par essayer d'unifier avec la condition \code{t(f) = t(e1)} (au cas où la définition n'est pas récursive). Il est apparu qu'on pouvait avoir des cycles. Mais en enlevant la contrainte \code{t(f) = t(e1)}, on peut unifier sans problème. De même, si c'est autorisé, on généralise juste après.
	
	\paragraph{Résultats} Pour ce qui est des résultats, le système de typage semble assez robuste, et fidèle à OCamL. Voici quelques exemples 
	
	\begin{minted}{ocaml}
	let rec f = f (f 7) in f
	\end{minted}
	
	est typé en \code{int -> int}
	
	\begin{minted}{ocaml}
	let a,b = (0+0, fun x -> x) in a, b
	\end{minted}
	
	est typé en \code{int * ('\_c -> '\_c)}. OCamL et Fouine s'interdisent de généraliser \code{'\_c -> '\_c}, à cause du \code{0 + 0} qui est une application de fonction, donc on ne généralise pas pour éviter d'avoir des problèmes.
	
	Autre exemple :
	\begin{minted}{ocaml}
	let f x y = x in f 5
	\end{minted}
	
	est typé en \code{'\_c -> int}, pareillement qu'en OCamL. Cependant, ce n'est pas le type le plus général ! on aurait envie d'avoir \code{'c -> int} ! mais :
	
	\begin{minted}{ocaml}
	let f x y = x in fun y -> f 5 y
	\end{minted}
	
	est bien typé en \code{'d -> int}. On généralise car on a une fonction à la racine. On arrive aussi à typer : 
	
	\begin{minted}{ocaml}
	let rec fix f = fun arg -> f (fix f) arg in fix
	\end{minted}
	
	en \code{(('h -> 'g) -> ('h -> 'g)) -> ('h -> 'g)}
	

\section{Pathologies}

	Dans cette partie, quelques pathologies sur le fonctionnement de fouine.
	
	
	\subsection{Types \code{unit -> 'a}}

	Le type \code{unit -> 'a} existe en OCamL par exemple avec
	\begin{minted}{ocaml}
					let f () = 5 ;;
	\end{minted}

	Ce code ne va pas passer en fouine car \code{()} ne correspond pas à un pattern de variables, or le constructeur pour les fonctions est \code{Fun of pattern\_f * expr\_f}. On utilise la variable "anonyme" \code{\_} pour cela :
	\begin{minted}{ocaml}
					let f _  = 5 ;;
	\end{minted}
	Ceci aura le comportement souhaité en pratique, car fouine ne vérifie pas les typages, et toutes les associations \code{\_ <- ...} sont ignorées par l'environnement. Ainsi, le type \code{unit -> 'a} devient \code{'a -> 'b}.

	\subsection{Fonctions récursives}
	
	On remarquera simplement que le code 
	
	\begin{minted}{ocaml}
					let f = 5 in
					let rec f = f;;
	\end{minted}
	
	va être accepté dans notre \emph{Fouine}, alors qui n'est pas sensé l'être en OCamL.
	Par contre, le code suivant ne l'est pas (et c'est normal, il ne l'est pas non plus en Ocaml) :
	
	\begin{minted}{ocaml}
					let rec f = fun x -> f;;
	\end{minted}

	\subsection{\code{;;}}
	
	Il me semble que le \code{;;} n'a pas le comportement désiré ...
	C'est une remarque à 22h le jour du rendu, donc qu'il en soit ainsi.
	
\end{document}

