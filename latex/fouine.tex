\documentclass{article}

\usepackage[utf8]{inputenc}
\usepackage[T1]{fontenc}
\usepackage[francais]{babel}
\usepackage{fullpage}


\usepackage{amsthm}
\usepackage{amsmath}
\usepackage{amssymb}
\usepackage{mathrsfs}

\title{\textbf{Fouine}}
\date{Mars 2018}
\author{Victor Boone - Gabrielle Pauvert}

\usepackage{graphicx}
\DeclareGraphicsExtensions{.png}
\usepackage{hyperref}

\newcommand\code[1]{{\fontfamily{lmtt}\selectfont #1}}

\usepackage{minted} 
 
\begin{document}

\maketitle
\tableofcontents

\section{Généralités}

	Dans notre implémentation de \emph{Fouine}, il y a $4$ grands types
	
	\vspace{0.5cm}
	
	\begin{tabular}{l l l}
	Composante & Description & Type \\
	\hline
	Expression & Ce qu'on évalue & \code{expr\_f} \\
	Valeurs & Ensemble des valeurs pouvant être renvoyées & \code{val\_f} \\
	Environnement & Stocke des couples (variables, valeurs) & \code{env\_f} \\
	Mémoire & Représentation de la mémoire en Fouine & \code{mem\_f} 
	\end{tabular}
	
	\vspace{0.5cm}
	
	On ne parlera pas de \emph{programme} en Fouine, mais plutôt d'\emph{expression}. Tout est expression qu'on cherche à évaluer. L'évaluation des fonctions est faite par la fonction \code{eval}:
	
	
	\begin{minted}{ocaml}
	
val eval : expr_f -> env_f -> (val_f -> val_f) -> (val_f -> val_f) list -> val_f
	
	\end{minted}
	
	utilisée de la forme \code{eval expr env k kE}, et implémentée par continuations.
	La mémoire \code{mem\_f} est globale.

\subsection{Expressions}

	Les expressions regroupent tout ce qui a été demandé dans le sujet. Leur type est :
	
	\begin{minted}{ocaml}
	
type expr_f =
  | Cst    of int                          (* Feuille : constante *)
  | Bool   of bool                         (* Feuille : booléen  *)
  | Var    of var_f                        (* Feuille : variable *)
  | Bang   of expr_f                       (* Feuille : le déréférençage *)
  | Unit                                   (* Feuille : le type unit *)
  | Pair   of expr_f * expr_f              (* Un couple d'expressions *)
  | Neg    of expr_f                       (* Négation de Booléens *)
  | Bin    of expr_f * operator_f * expr_f (* opérations binaires *)
  | PrInt  of expr_f                       (* built-in prInt *)
  | Let    of pattern_f * expr_f * expr_f  (* let <var_f> = <expr_f> in <exec_f>   *)
  | LetRec of var_f   * expr_f * expr_f    (* let rec *)
  | Match  of expr_f  * pmatch_f           (* match [expr_f] with [pattern_matching] *)
  | IfElse of expr_f * expr_f * expr_f     (* If .. then .. else *)
  | Fun    of pattern_f * expr_f           (* car les fonctions sont un objet fun var -> expr *)
  | App    of expr_f * expr_f              (* Ce sont les applications *)
  | Aff    of expr_f * expr_f              (* Affectation i.e le `:=`*)
  | Alloc  of expr_f                       (* Allocation mémoire *)
  | Try    of expr_f * var_f * expr_f      (* Le 'try ... with E ... -> g...' *)
  | Raise  of expr_f                       (* raise E ... : qui sera un int en pratique *)


	\end{minted}
	
	\vspace{0.5cm}
	Voici un tableau de correspondance entre les constructeurs et les notions OCamL.
	
	\begin{center}
	\begin{tabular}{l | l}
		Constructeur & Équivalent OCamL \\
		\hline
		\code{Var of var\_f} & \code{x, y, c0, variable\_1} : nom de variable \\
  		\code{Bang of expr\_f} & \code{!} : déréférençage\\
  		\code{Bool of bool} & \code{true}, \code{false} : les booléens\\
		\code{Cst of int} & \code{0}, \code{1} : les entiers\\
		\code{Neg of expr\_f} & \code{not ...} : négation booléenne \\
		\code{Bin of expr\_f * operator\_f * expr\_f} & \code{x+5} : opérations binaires (\code{+, -, *, /, mod, ||, <, =, ...}) \\
  		\code{PrInt of expr\_f} & \code{let prInt x = print\_int x; print\_newline ()} \\
		\code{Let of pattern\_f * expr\_f * expr\_f} & \code{let ... = ... in ... } \\
		\code{LetRec of var\_f * expr\_f * expr\_f} & \code{let rec ... = ... in ... }\\
		\code{IfElse of expr\_f * expr\_f * expr\_f} & \code{if ... then ... else ...} \\
		\code{Fun of pattern\_f * expr\_f} & \code{fun x -> ... } \\
		\code{App of expr\_f * expr\_f} & \code{ a b } : application \\
		\code{Aff of expr\_f * expr\_f} & \code{... := ...} \\
  		\code{Alloc of expr\_f} & \code{ref ...} : allocation mémoire \\
  		\code{Pair of expr\_f * expr\_f} & \code{... , ...} : couples \\
  		\code{Raise of expr\_f } & \code{raise (E ...) } : levée d'exceptions \code{E}\\
  		\code{Try of expr\_f * var\_f * expr\_f} & \code{try ... with E x -> ...} 
	\end{tabular}
	\end{center}
	
	\vspace{0.5cm}
	
	Certains constructeurs intermédiaires sont détaillés dans \code{type.ml}.
	Les booléens font leur apparition dans les expressions de ce \emph{Fouine}, principalement pour simplifier les transformations de programmes. \code{Bin} a été étendu en conséquence.

\subsection{Valeurs}


	Les programmes fouine sont des expressions \code{expr\_f}, et sont évaluées par la fonction \code{eval}. Une évaluation renvoie une valeur du type donné ci-dessous :
\begin{minted}{ocaml}
type val_f = Unit
          | Bool        of bool
          | Int         of int
          | Ref         of int
          | Cons        of string * val_f
          | Pair_val    of val_f * val_f
          | Fun_val     of pattern_f * expr_f * env_f
\end{minted}

	Concernant les fonctions, celles-ci sont de la forme \code{fun x -> expr}. On sauvegarde de plus l'environnement dans lequel elles ont été définies (notion de clotûre). Nous y reviendrons. \code{Cons} n'est pas utilisé.


\subsection{Environnement}

	L'environnement est une liste d'association (variable, valeur), qui agit comme une \textbf{pile} (on empile les associations les unes après les autres). Une variable est simplement une chaîne de caractères. Ainsi :
	
	\begin{minted}{ocaml}
		type env_f = (var_f * val_f) list
	\end{minted}
	
	Si \code{x} est une variable, sa valeur associée dans un environnement \code{env} est la première occurrence \code{("x", ...)} dans l'environnement. Donc, si on considère la liste \code{l = [("x", Int(0)); ("x", Int(5))]}, la valeur de \code{x} courante est \code{Int(0)}.
	
	Si on essaie de lire la valeur d'une variable non-existante, l'interpréteur lève une exception \code{Failure}.

\subsection{Mémoire}

	
	La mémoire est un tableau de valeurs de taille fixe \code{1000000}. Ainsi, on impose à \emph{Fouine} une quantité bornée de mémoire. Si celle-ci est dépassée, l'interpréteur lève \code{Failure "Out of Memory"}. Les cases mémoires étant typées \code{val\_f}, elles peuvent occuper une taille non bornée de mémoire.
	
	\begin{minted}{ocaml}
let mem = Array.make 1000000 Unit
let available = ref 0;;
	\end{minted}
	
	\vspace{0.5cm}

	C'est la mémoire qui gère les références. Pour les gérer, on descend assez bas-niveau dans la philosophie. Les références seront vues comme une adresse mémoire. Ainsi, \code{let a = ref 0} est interprété comme

	\vspace{0.5cm}

\begin{itemize}
	\item Allouer une nouvelle case mémoire
	\item Associer \code{a} à cette nouvelle case \emph{(notion d'adresse mémoire)}
	\item Mettre le contenu de cette case à $0$
\end{itemize}

	\vspace{0.5cm}
	
	Pour savoir qu'elle est la prochaine case mémoire allouable, on utilise une variable globale \code{available}.

\section{Parsing}


\subsection{\code{let ... in}}

Ce qui pose problème avec les \code{let ... in ...} c'est qu'il y en a beaucoup de variantes! On distingue les \code{let} des expressions des \code{let} extérieurs (du toplevel) qui permettent une syntaxe un peu particulière pour enchaîner les \code{let} sans \code{in} ni \code{;;}. Il y a aussi l'ajout (parfois facultatif) du mot clé \code{rec} qui dédouble tous les cas. Enfin les \code{let} peuvent servir à définir des fonctions ou des couples (triplets, etc.). Les let ont été un peu regroupés depuis le rendu 3. Une variable simple est vue comme un cas particulier de uplet à un élément. On définit une règle \code{var\_pattern} qui admet tous ces cas. Les \code{let rec} ne sont pas concernés: on ne peut définir qu'au plus une variable avec un \code{let rec}.

\subsection{L'application de fonction}

Typiquement, une application de fonctions est du type \code{expression expression}, mais écrire cela en tant que règle aurait abouti à du "rule never reduce" car on peut toujours lire une expression derrière, et choisir de \code{shift}. Le plus simple était d'expliciter tous les cas dans une règle \code{applicator applicated} où \code{applicator} sert à enchaîner les applications (curryfication) et \code{applicated} détaille les possibilités de fonctions et d'arguments.

Pour la définition de fonction, il a fallu faire attention aux différentes façons de définir une fonction (avec des \code{fun ->} ou directement, ce qui complique encore un peu les définitions de \code{let... in...}).


\section{Evaluation}

\subsection{Style d'évaluation}

	L'évaluation se fait par continuation : \code{eval expr env k kE}. La fonction d'évaluation reçoit quatre arguments :

	\begin{itemize}
		\item \code{expr} l'expression à évaluer
		\item \code{env}  l'environnement courant (définition des variables)
		\item \code{k}	  la continuation courante
		\item \code{kE}	  la pile de continuations d'exceptions	
	\end{itemize}
	
	Nous renvoyons au code pour l'implémentation de \code{eval}. Certains passages du code sont plus lourd car \code{eval} est capable d'afficher l'expression courante en cas d'erreur - l'implémentation de l'évaluation fainéante a aussi été gourmande en lignes de code.
	
\subsection{Fonctions}

	\subsubsection{Fonctions classiques}

	Les fonctions sont définies avec une clôture pour la raison suivante 
	
	\begin{minted}{ocaml}
let a = 5 in
let f = fun x -> a in
let a = 10 in
f 10;;
	\end{minted}
	
	Ici, \code{f} doit se souvenir de la valeur de \code{a}. Sauvegarder une clôture est en ${\cal{O}}(1)$ : il s'agit juste de sauvegarder un pointeur vers une tête de pile i.e l'environnement courant. La valeur d'une fonction est 
	
	\begin{minted}{ocaml}
	
	| Fun_var of var_f * expr_f * env_f 
	
	\end{minted}
	
	donc les fonctions ne sont pas typées en fouine.
	
	\subsubsection{Fonctions Récursives}	

	Quand on définit une fonction \code{f} de la manière ci-dessus, il faut remarquer qu'au moment de sa définition, \code{f} n'est pas définie dans l'environnement, donc une fonction standard n'est pas définie dans sa propre clôture. On ne peut donc pas définir des fonctions récursives. On introduit le \code{let rec}.
	
	\vspace{0.5cm}
	
	Le \code{let rec} est un peu particulier. Il regarde si on est en train de définir une fonction ou non. Si oui, il construit la clôture de \code{f} en y rajoutant l'association \code{(f, clôture où f est défini)}. Il y a donc une définition cyclique ici, qui fera apparaître \code{<cycle>}. Un détail important ; lors d'un
	
	\begin{minted}{ocaml}
let rec f = e1 in e2
  \end{minted}
  
  on commence par évaluer \code{e1}. Il y a alors deux possibilitées. Soit on reçoit une valeur fonctionnelle \code{Fun\_val("f", e0, env0)}, avec \code{env0} une clôture dans laquelle \code{f} n'est pas défini. C'est là qu'on utilise le \code{let rec} d'OCamL pour redéfinir la clôture de la fonction avant de rajouter \code{f} dans l'environnement. On évalue ensuite \code{e2}.
	
	\vspace{0.5cm}
	
	Si c'est une variable quelconque, qui n'est pas une fonction, on se comporte comme un \code{let} classique. Ceci ne pose pas trop de problèmes si on ne joue qu'avec des \code{int} ou des couples, mais fait tout de même apparaître quelques petits cas pathologiques (cf dernière partie).



\section{Exceptions}

  \code{eval} est implémenté par continuations, et dispose d'une pile de continuations d'exceptions. En cas de \code{try e1 with e2} l'interpréteur ajoute une nouvelle continuation d'exception sur la pile pour l'évaluation de \code{e1}. Quant à \code{raise e1}, il évalue \code{e1}, puis dépile la pile de continuations d'exceptions pour traiter le résultat.
  
  \vspace{0.5cm}
  
  \code{Remarque :} On pourrait se débarrasser de la pile, pour n'avoir qu'une continuation d'exception à chaque fois en changeant 
  
  \begin{minted}{ocaml}
| Raise expr ->
		begin
		  match k' with
		  | [] -> failwith "Raise : Nothing to catch exception"
		  | k_exn :: k' -> eval expr env k_exn k'
    end
| Try (expr1, var_except, expr2) ->
   eval expr1 env k ((fun exn -> eval expr2 (env_aff var_except exn env) k k') :: k')
  \end{minted}
  
  en
  
  \begin{minted}{ocaml}
| Raise expr ->
	 eval expr1 env k' k'
| Try (expr1, var_except, expr2) ->
   eval expr1 env k (fun exn -> eval expr2 (env_aff var_except exn env) k k')
   \end{minted}
  
  Ceci nous a surpris à première vue, mais c'est ce qu'on a fait pour les transformations de programme et cela semble très bien marcher.

\newpage
\section{Transformations de Programmes}

	On met ici les formules utilisées pour les transformations de programmes. Ces dernières ont été converties en expressions d'arbres de programme en utilisant directement le \textit{parser}.
	
\subsection{Impératives}

	On rajoute en pratique un \code{\_\_} aux variables utilisées pour différencier de celles du programme initial.
	
	\begin{minted}{ocaml}

[| 42 |] (* devient *) fun s -> (42,s)
[| true |] (* devient *) fun s -> (true,s)
[| x |]  (* devient *) fun s -> (x,s)
[| !e |] (* devient *) fun s -> let (l,s1) = [| e |] s in
                       let v = read s1 l in (v,s1)
[| () |] (* devient *) fun s -> ((),s)

[| (e1, e2) |] (* devient *) 
  fun s -> let (v2,s2) = [| e2 |] s in
           let (v1,s1) = [| e1 |] s2 in
           ((v1,v2),s1)

[| not e0 |] (* devient *) 
  fun s -> let (b,s0) = [| e0 |] in
           (not b, s0)
[| e1 + e2 |] (* devient *) 
  fun s -> let (v2,s2 ) = [[e2]] s in
           let (v1,s1) = [[e1]] s2 in
           (v1 op v2,s1)

[| prInt e0 |] (* devient *) 
  fun s -> let (v0,s0) = [| e0 |] s in
           (prInt v0, s0)
[| let pat = e1 in e2 |]   (* devient *) 
  fun s -> let (pat,s1) = [| e1 |] s in
           [| e2 |] s1
[| let rec v = e1 in e2 |] (* devient *) 
  fun s -> let rec v = (let (f,s0) = [| e1 |] s in f)
           in [| e2 |] s

[| match |] (* non supporté *)
[| if b then e1 else e2 |] (* devient *) 
  fun s -> let (b0, s0) = [| b |] s in
           if b0 then [| e1 |] s0
           else [| e2 |] s0
           
[| fun pat -> e |] (* devient *)  fun s -> ((fun pat -> [| e |]), s)
[| e1 e2 |] (* devient *) 
  fun s -> let (v, s2) = [| e2 |] s  in
           let (f, s1) = [| e1 |] s2 in
           f v s1

[| e1 := e2 |] (* devient *) 
  fun s -> let (l1, s1) = [| e1 |] s in
           let (v2, s2) = [| e2 |] s2 in
           let s3 = write s2 l1 v2 in
           ((), s3)
[| ref e0 |] (* devient *) 
  fun s -> let (v,s1) = [| e0 |] s in
           let (s2,l) = alloc s1 in
           let s3 = write s2 l v in
           (l,s3)
           
[| try e1 with E x -> e2 |] (* devient *) 
  fun s -> try let (v1, s1) = [| e1 |] s in (v1, s1)
           with E x -> [| e2 |] s
           
[| raise e0 |] (* devient *) 
  fun s -> try let (v,s0) = [| e0 |] s in
	       (raise (E v), s0)
	
	\end{minted}
	

\subsection{Continuations}

	On rajoute en pratique un \code{\_} aux variables utilisées pour les différencier de celles du programme initial.

	\begin{minted}{ocaml}

[| () |] (* devient *) fun k kE -> k ()
[| true |] (* devient *) fun k kE -> k true
[| 42 |] (* devient *) fun k kE -> k 42
[| x |]  (* devient *) fun k kE -> k x

[| !e |] (* devient *) 
	fun k kE -> [| e |] (fun addr -> k (!addr)) kE

[| not e |]   (* devient *) 
	fun k kE -> [| e |] (fun b -> k (not b)) kE
[| e1 + e2 |] (* devient *) 
	fun k kE -> [| e2 |] (fun v2 -> [| e1 |] (fun v1 -> k (v1 + v2)) kE) kE
[| prInt e |] (* devient *) 
	fun k kE -> [| e |] (fun v -> k (prInt v)) kE

[| if be then e1 else e2 |] (* devient *) 
	fun k kE -> [| be |] (fun b -> (if b then [| e1 |] else [| e2 |]) k kE) kE
	
	\end{minted}
	
	C'est là qu'il est agréable de supporter les booléens, car \code([| be |]) renvoie une valeur booléenne. Il est possible d'esquiver ce problème autrement, mais de manière moins fluide.
	
	\begin{minted}{ocaml}
	
[| let x = e1 in e2 |] (* devient *) 
	fun k kE -> [| e1 |] (fun v -> let x = v in [| e2 |] k kE) kE 
[| let rec f = e1 in e2 |] (* devient *) 
	fun k kE -> [| e1 |] (fun v -> let rec f = v in [| e2 |] k kE) kE

	\end{minted}
	
	Ici, on utilise le fait que \code{[| e1 |]} renvoie une valeur fonctionnelle, mais non récursive. L'utilisation d'un \code{let rec} juste derrière permet de redéfinir cette valeur fonctionnelle comme étant récursive. Ceci fonctionne car \emph{Fouine} n'évalue pas les fonctions, mais les renvoie directement avec leur clôture.
	
	\begin{minted}{ocaml}
	
[| match |] (* non supporté *)
[| fun x -> e |] (* devient *) 
	fun k kE -> k (fun x -> [| e |]) 
[| e1 e2 |] (* devient *) 
	fun k kE -> [| e2 |] (fun v -> [| e1 |] (fun f -> f v k kE) kE) kE

[| e1 := e2 |] (* devient *) 
	fun k kE -> [| e2 |] (fun v -> [| e1 |] (fun addr -> k (addr := v)) kE) kE
[| ref e |]    (* devient *) 
	fun k kE -> [| e |] (fun v -> k (ref v)) kE

[| (e1, e2) |] (* devient *) 
	fun k kE -> [| e2 |] (fun v2 -> [| e1 |] (fun v1 -> k (v1, v2)) kE) kE	

[| try e1 with E x -> e2 |] (* devient *) 
	fun k kE -> [| e1 |] k (fun x -> [| e2 |] k kE)
[| raise e |]               (* devient *) 
	fun k kE -> [| e |] (fun v -> kE v) kE
	
	
	\end{minted}


\section{La machine SECD}

\subsection{Le langage de la machine}



\subsection{La transformation}

\subsection{L'exécution}


\section{Pathologies}

	Dans cette partie, quelques pathologies sur le fonctionnement de fouine.
	
	
	\subsection{Types \code{unit -> 'a}}

	Le type \code{unit -> 'a} existe en OCamL par exemple avec
	\begin{minted}{ocaml}
					let f () = 5 ;;
	\end{minted}

	Ce code ne va pas passer en fouine car \code{()} ne correspond pas à un pattern de variables, or le constructeur pour les fonctions est \code{Fun of pattern\_f * expr\_f}. On utilise la variable "anonyme" \code{\_} pour cela :
	\begin{minted}{ocaml}
					let f _  = 5 ;;
	\end{minted}
	Ceci aura le comportement souhaité en pratique, car fouine ne vérifie pas les typages, et toutes les associations \code{\_ <- ...} sont ignorées par l'environnement. Ainsi, le type \code{unit -> 'a} devient \code{'a -> 'b}.

	\subsection{Fonctions récursives}
	
	On remarquera simplement que le code 
	
	\begin{minted}{ocaml}
					let f = 5 in
					let rec f = f;;
	\end{minted}
	
	va être accepté dans notre \emph{Fouine}, alors qui n'est pas sensé l'être en OCamL.
	Par contre, le code suivant ne l'est pas (et c'est normal, il ne l'est pas non plus en Ocaml) :
	
	\begin{minted}{ocaml}
					let rec f = fun x -> f;;
	\end{minted}
	
	\subsection{Références}
	
	\begin{minted}{ocaml}
				let a = ref (ref 42) in
				a := ref (ref 4);
				prInt !(!(!a));;
	\end{minted}
	
	Ici, on a un code fouine accepté, qui ne l'est pas en OCamL, car OCamL perçoit une erreur de typage.

\end{document}

